\chapter{Related Work}

A lot of effort has been put in the last decades to develop efficient algorithms and tools for converting a triangle-mesh into a quad-mesh (also known as \emph{mesh quadrangulation}). Since there is a detailed and nice survey on quad-mesh generation and processing \cite{10.1111/cgf.12014}, we will give a very brief review of the two main families of quadrangulation techniques.

\section{Quad Conversion}
\emph{Quad Conversion} is a type of method which directly converts a triangle-mesh into a quad-mesh by explicitly manipulating the mesh's elements connectivity. A naive method to convert any polygonal mesh into a quad-mesh is to perform a single iteration of Catmull-Clark subdivision \cite{Catmull1978RecursivelyGB}. The resulting mesh is a quad-mesh by definition. However, it comes at a price of great increase to the mesh complexity. Another approach is based on pairing two adjacent triangles on the input triangle mesh, and fusing them together into a quadrangle on the output quad-mesh. Since this approach gradually changes the mesh's connectivity by applying a local operator, this approach generates an unstructured quad-mesh. Also, since this type of methods do not use any surface curvature information, is produce an output mesh which doesn't capture the mesh's geometric feature lines. A brief review of methods of this type are available in \cite{10.1111/cgf.12014}.

\section{Surface Patching}
\emph{Surface Patching} is a type of methods that maps the mesh's surface onto a set of rectangular euclidean 2D patches. By tessellating each patch with quads, the parametrization of the mesh's surface using the quadrangulated patches is trivial, and a valid quad-mesh can easily be extracted. This type of methods produce semi-regular quad-meshes, where irregular vertices are located at stitching points of the patches on the mesh's surface. However, they are also known to introduces angle distortion and more irregular vertices in the extracted mesh.

\section{Quad Remeshing}
\emph{Quad Remeshing} is a type of methods which involve resampling of the input mesh surface during the quad-mesh conversion process. Methods of this kind usually produce valence semi-regular quad-mesh as the output mesh, which preserve the geometric features of the original mesh by exploiting quadrangulation guiding-fields such as the principal curvature directions. Generally, there are two classes of quad remeshing methods:
\begin{enumerate}
\item Direct Parametrization Methods
\item Field guided Methods
\end{enumerate}
We elaborate on each type of method in the sub-sections below.

\subsection{Direct Parametrization Methods}
In \emph{Direct Parametrization Methods}, the input triangle-mesh is cut by a set of seams (a seam is defined by a series of consecutive adjacent edges) into an open 2D surface homeomorphic to a disk, embedded in 3D ambient space. Once the mesh is open, a mapping $\phi:\mathcal{M}^2\rightarrow\mathbb{R}^2$ is defined over the 2D manifold to copy each triangle onto the 2D Euclidean domain/plane. Since the 2D Euclidean plane can be trivially tasselled into regular and uniform quads by the canonical Cartesian grid, lifting the grid's isolines back onto the mesh's manifold, by utilizing the inverse mapping $\phi^{-1}$, forms a valid quad-mesh on the input mesh's 3D surface (such that the grid's isolines are correctly stitched together), given that mapping $\phi$ satisfies the set of necessary and sufficient conditions of Integer Grid Maps \cite{bommes:hal-00862648}.

\subsection{Field Guided Methods}
These types of method, which seem to be the most popular, separates the process of quadrangulating a triangle-mesh into three independent phases:

\begin{enumerate}
\item Orientation field generation.
\item Sizing field generation.
\item Quad-mesh extraction.
\end{enumerate}

\paragraph{Orientation Field} In this phase, a cross field, which is composed of four coupled vector fields, is associated with the faces of the input triangle mesh \cite{10.1145/1356682.1356683}. By providing a set of initial constraints, a smooth cross field is interpolated across the triangle-mesh's surface. The initial constraints can be given directly by the user (e.g. by utilizing brush tools) or by heuristics based on principal curvature directions.

\paragraph{Sizing Field} The sizing field determines the side-length of the generated quads, by controlling the spacing between the parameterized Cartesian grid's isolines on the mesh's surface. The sizing field can be anisotropic (rectangular quads) or isotropic (square quads). In many cases, a constant sizing field is sufficient and/or required. A dynamic sizing field, which varies in correlation with the curvature, is useful in order to better approximate the geometric features of the input mesh's surface.

\paragraph{Quad-Mesh Extraction} This is the last phase, where the orientation and sizing fields are used to map the Cartesian grid's isolines on the input triangle-mesh's surface. In order for the isolines to be aligned with orientation field, an optimization problem is formulated using a suitable energy function to search for a bi-variate mapping from the $\left(u,v\right)$ parametrization space to the output mesh's surface, such that its gradient at each point is aligned with the orientation field, and it also satisfied the sufficient and necessary conditions of integer grid maps, thus allowing to extract a pure quad-mesh. An alternative way to extract the output mesh, as been done by \cite{10.1145/882262.882296} and \cite{10.5555/1025128.1026044} is to use an \emph{Explicit Extraction Method}, where the mesh is streamlined with curves that follows the orientation field, and spaced with the required sizing field. The streamlines curves are sampled, and the mesh polygons are extracted.

\par
\noindent Multiple field-guided quad remeshing approaches have been proposed in the last decade. In \cite{10.1145/1531326.1531383}, an orientation field is automatically interpolated over the input mesh's surface using a sparse set of constraints which can be given by the user and/or extracted from geometric features of the original mesh where the principal curvature directions are clear enough (feature points with parabolic nature). Next, a mixed-integer optimization problem is solved to smooth the interpolated orientation field, and  finally a second mixed-integer optimization problem is solved in order to find a global parametrization integer grid map which is used to extract the output quad-mesh. In \cite{10.1145/2816795.2818078}, they take a local approach (in contrast to the global approach in \cite{10.1145/1531326.1531383}) both for building and smoothing the orientation field over the input mesh's surface, and for finding the integer grid map that parametrizing the 2D Cartesian grid over the input mesh's surface, by exploiting the extrinsic properties of the guiding field, and not the intrinsic properties. This approach allow their method to snap to sharp geometric features of the input mesh without any initial constraints. In contrast to \cite{10.1145/2816795.2818078}, which produce pure quad-meshes with much fewer singularities at the cost of longer computation times due to the global parametrization approach, the output mesh produced by \cite{10.1145/2816795.2818078} is a quad-dominant mesh, which is composed mostly of quads, but also of triangles and pentagons, and is prone to higher number of singularities due to the local optimization approach. In \cite{10.1111:cgf.13498}, they modify the original algorithm given by \cite{10.1145/2816795.2818078} such that it efficiently produces meshes with fewer singularities. They propose an efficient method to minimize singularities by combining the objective given by \cite{10.1145/2816795.2818078} with a system of linear and quadratic constraints.

\par
\noindent The main virtue of field guided methods is that the challenging problem of quadrangulating a triangle-mesh is divided into a set simpler sub-problems, of which each can be solved independently with the best suited element representations and algorithms. On the other hand, this approach complicates the implementation of the quadrangulation process. In our work, we take a direct parameterization approach which attempts to quadrangulate a triangle-mesh in a single phase, by solving a smooth and unconstrained optimization problem directly, and therefore, achieving a simpler implementation which can be more easily adapted and extended with additional requirements and objectives. 