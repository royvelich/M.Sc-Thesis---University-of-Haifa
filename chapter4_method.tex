\chapter{Method}
At its core, our method formulates the necessary and sufficient conditions for quadrangulating a triangle-mesh by parametrization (integer-grid maps \cite{bommes:hal-00862648}), as a smooth unconstrained optimization problem using periodic functions. Our main objective function is a weighted sum of smooth sub-objective functions, of which their weights are controllable by the user. We believe that this approach is more suitable for 3D design and modeling tools since the quadrangulation process is continuous with the user constantly in the loop. This approach is also easily extensible by introducing additional sub-objectives smooth terms into the main objective, which can represent custom requirements of which the parameterization has to fulfill. Moreover, since the weights are controllable, the quadrangulation process can be executed in any desired order (e.g. achieve a seamless parametrization before handling singular points). That's why we decided  to call our approach \emph{Smooth and Interactive Parametrization-Based Quadrangulation}: \emph{Smooth} because we're using smooth objective, \emph{Interactive} because small changes are immediately reflected to the user, and \emph{Parametrization-Based} because we solve a global parametrization problem. This chapter is organized as following: In section \ref{integer-grid-maps} we describe in detail the notion of inter-grid maps. In section \ref{label:quad_distortion_cond} we briefly describe how to we cope with quad-distortion in order to maximize mesh quality.

\section{Integer Grid Maps}
\label{integer-grid-maps}
The necessary and sufficient conditions for a parametrization mapping to form a valid quad mesh on the mesh's 3D surface are:
\begin{enumerate}
\item Seamless Condition
\item Singular Points Condition
\item Consistent Orientation Condition
\end{enumerate}

\subsection{Seamless Condition}
\label{label:seamless_cond}
The transition function $g_{ij}$ between two half-edges $e_i$ and $e_j$ on the parameterization domain that corresponds to the same surface edge that is part of a cut seam, has to be an integer-grid automorphism given by:
\begin{equation}\label{transition_g_ij}
\begin{split}
e_j = R^{r_{ij}}_{90^\circ}e_i + \vec{t}_{ij}
\end{split}
\end{equation}
Where  $r_{ij} \in \{0,1,2,3\}$ and $\vec{t}_{ij} \in \mathbb{Z}^2$. Figures \ref{fig:translation_req}, \ref{fig:angle_req} and \ref{fig:length_req} visually demonstrate the 3 requirements encoded by the seamless condition.
  
\begin{figure}[ht]
\centering
\includegraphics[width=9cm]{figures/seamless/translation.png}
\caption[The Translation Requirement]{The transition function $g_{ij}$, described by equation \ref{transition_g_ij}, should impose an integer translation between the two half edges.}
\label{fig:translation_req}
\end{figure}

\begin{figure}[ht]
\centering
\includegraphics[width=9cm]{figures/seamless/angle.png}
\caption[The Angle Requirement]{The transition function $g_{ij}$, described by equation \ref{transition_g_ij}, should impose a $\frac{\pi}{2}k$ rotation between the two half edges.}
\label{fig:angle_req}
\end{figure}

\begin{figure}[ht]
\centering
\includegraphics[width=9cm]{figures/seamless/length.png}
\caption[The Length Requirement]{The transition function $g_{ij}$, described by equation \ref{transition_g_ij}, should implicitly match the length of the two half edges.}
\label{fig:length_req}
\end{figure}

\subsection{Singular Points Condition}
\label{label:singular_points_cond}
All singular vertices, which are characterized by a non-zero angular defect on the parameterization domain, have to lie on integer locations. That is, given the set $S_i$ of all parameterization domain vertices that correspond to the same singular surface vertex $v_i$, we require that:
$$\forall u \in S_i: u \in \mathbb{Z}^2 $$
Figure \ref{fig:singular_points_req} demonstrate this condition visually.
  
\begin{figure}[ht]
\centering
\includegraphics[width=9cm]{figures/singular_points/singularity.png}
\caption[The Singular Points Requirement]{All of the domain vertices that correspond to the same singular point should be positioned at a grid point (integer location).}
\label{fig:singular_points_req}
\end{figure}

\subsection{Consistent Orientation Condition}
\label{label:consistent_otrientation_cond}
All triangles on the parameterization domain should have the same orientation. In other words, the transition function should not allow triangle flips. Figure \ref{fig:orientation_req} demonstrate this condition visually.

\begin{figure}[ht]
\centering
\includegraphics[width=9cm]{figures/orientation/orientation.png}
\caption[The Orientation Requirement]{The left triangle follows the same counter clockwise orientation as its image on the mesh's surface. On the other hand, the vertices of the triangle on the right follows a clockwise orientation.}
\label{fig:orientation_req}
\end{figure}

\section{Quad Distortion}
\label{label:quad_distortion_cond}
The fulfilment of the integer-grid maps conditions do guarantee that the isolines of the 2D Cartesian grid forms a valid quad-mesh on the 3D mesh's surface, however, we do have to consider at what cost. There are many possible integer-grid maps that transform a given triangle-mesh into a quad-mesh. Amongst them, we would like to pick the one that satisfies our desired set of additional requirements. In order to meet this desiderata, we have to pay with shape distortion of the triangles that make up the parametrization domain. Ultimately, we would like to find a map that fits our needs and minimizes the triangles distortion, since a distorted triangle renders a distorted isoline on the input mesh surface, which eventually affects the quality of the extracted quads that correspond to that triangle. Therefore, in addition to the integer-grid maps conditions, we add an additional implicit core requirement, which demands to find a mapping that minimizes triangle distortion. Figure \ref{fig:distortion_req} demonstrate this requirement visually.

\begin{figure}[ht]
\centering
\includegraphics[width=9cm]{figures/distortion/distortion.png}
\caption[The Orientation Requirement]{The left triangle has a little shape distortion compared with its original version on the input-mesh's surface, in contrast with the right triangle, which clearly express a high shape distortion compared with its original version}
\label{fig:distortion_req}
\end{figure}

\section{Smooth Objective Functions}
In this section, we present our smooth objective function formulations for Conditions \ref{label:seamless_cond}, \ref{label:singular_points_cond}, \ref{label:consistent_otrientation_cond} and requirement \ref{label:quad_distortion_cond} presented in the previous sections.

\subsection{Seamless Condition Objective Functions}
The seamless condition can be expressed by three smooth objective functions $P_{angle}$, $P_{length}$ and $P_{translation}$, that penalize violations in angle, length and translation, respectively, for a given pair of half-edges.

\paragraph{Angle Penalty}
Given two half-edges $e^i = \left(p^i_1,p^i_2\right)$ and $e^j = \left(p^j_1,p^j_2\right)$ in the parametrization domain, we define $P_{angle}$ as follows:

\begin{equation}\label{angle_penalty}
\begin{split}
P_{angle}\left(e^i,e^j\right) = \mathrm{sin} \bigg( 4\Big(\theta\left(e^i\right) - \theta\left(e^j\right)\Big) - \frac{\pi}{2}\bigg) + 1
\end{split}
\end{equation}
Where given that $k\in\left\{i,j\right\}$, $v^k_1$ and $v^k_2$ represent the first and second vertices, respectively, that constitutes half-edge, $\theta\left(e^k\right) = \mathrm{atan2}\Big(y\left(v^k_2\right) - y\left(v^k_1\right), x\left(v^k_2\right) - x\left(v^k_1\right)\Big)$ is the angle formed by half-edge $k$ with the positive \emph{x-axis} direction, and $x\left(v\right)$, $y\left(v\right)$ return the $x$ and $y$ coordinates, respectively, of an input vertex $v$. Please note that $\theta\left(e^i\right) - \theta\left(e^j\right)$ yields the signed angle between the two half-edges. Figure \ref{fig:angle_penalty} shows a plot of the angle penalty function. It is easy to see that the angle penalty is minimized when the angle discrepancy between the two half-edges if an integer multiple of $\frac{\pi}{2}$, as required by the seamless condition.

\begin{figure}[ht]
\centering
\includegraphics[width=15cm]{figures/seamless/angle_penalty_sin.png}
\caption[Angle Penalty]{A plot of the angle penalty function. The penalty is zero for half-edges with angle discrepancy of integer multiples of $\frac{\pi}{2}$}
\label{fig:angle_penalty}
\end{figure}